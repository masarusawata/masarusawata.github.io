\documentclass[letterpaper, 12pt]{article}
\usepackage{amsmath}
\usepackage[margin=1in]{geometry}
\usepackage{adjustbox}
\usepackage{graphicx}
\usepackage[final]{pdfpages}
\usepackage{bm}
\usepackage{sectsty}
\usepackage{titlesec}
\usepackage{lipsum}
\usepackage{subcaption}
\usepackage{listings}
\usepackage{pgffor}
\usepackage{rotating}
\usepackage{xcolor}  
\usepackage{amsthm}
\usepackage{csvsimple}
\usepackage{enumitem}
\usepackage{amssymb} 
\usepackage{amsfonts}
\usepackage{tikz}
\usepackage{tikz-3dplot}
% \setlength{\tabcolsep}{0.5cm} 
\newcommand{\thickhat}[1]{\mathbf{\hat{\text{$#1$}}}}
\renewcommand{\lstlistingname}{Program}

\newtheoremstyle{custom}
  {3pt}
  {3pt}
  {\itshape}
  {} 
  {\bfseries}
  {. }  
  { }   
  {\thmname{#1} \thmnumber{#2} \thmnote{ #3}}

\theoremstyle{custom}


\newtheorem{definition}{Definition}
% \newtheorem{theorem}{Theorem}
\newtheorem*{theorem}{Theorem}
\sectionfont{\fontsize{12}{15}\selectfont}
\titleformat{\section}
{\normalfont\normalsize\bfseries}
{(\thesection)}{1em}{}

\title{An Elegant Method for Incorporating the Wellbore Storage Effect in the Laplace Domain}
\author{Masaru Sawata}
\begin{document}
\maketitle
There is an interesting technique for incorporating the wellbore storage effect into the constant-rate analytical solution.
This is one of my favorite topics in pressure transient analysis.

If we have a constant (sandface) rate solution $p_{wDc}(t_D)$ (dimensionless pressure), which may or may not include skin effect,
we can derive an analytical solution for a variable-rate problem by applying the principle of superposition.

First, we discretize the sand face rate:
\begin{equation*}
  p_{wD} \approx \frac{1}{q} \sum_{i=1}^{n} (q_{sf,i} - q_{sf,i-1}) p_{wDc} \left( t_{D} - t_{D,i-1} \right)
\end{equation*}
where $p_{wD}$ is the dimensionless pressure  that accounts for wellbore storage, $q$ is the (constant) surface rate, and $q_{sf}$ is the sand face rate.
Here, the subscript $i$ denotes denotes discretized time steps.

We define the dimensionless rate as $q_D = \displaystyle \frac{q_{sf}}{q}$, so the expression becomes:
\begin{equation*}
  p_{wD} \approx \sum_{i=1}^{n} (q_{D,i} - q_{D,i-1}) p_{wDc}\left( t_{D} - t_{D,i-1} \right)
\end{equation*}
\begin{equation*}
  p_{wD} \approx \sum_{i=1}^{n} \frac{q_{D,i} - q_{D,i-1}}{t_{D,i} - t_{D,i-1}} \left[ p_{wDc}\left( t_{D} - t_{D,i-1} \right) \right] \left( t_{D,i} - t_{D,i-1} \right)
\end{equation*}

If we take the limit $n \rightarrow \infty$ (divide the interval so that the $q_{D,i} - q_{D,i-1}$ becomes infinitesimaly small), 
\begin{equation*}
  p_{wD} = \int_{0}^{t_D} \frac{d}{d \tau}q_D(\tau) p_{wDc}(t_D - \tau) \, d\tau
\end{equation*}
This is a convolution integral.  Therefore, in the Laplace domain:
\begin{equation}
  \label{eq1}
  \mathcal{L} \left[ p_{wD} \right] = s \mathcal{L} \left[ q_D \right] \mathcal{L} \left[ p_{wDc} \right]
\end{equation}
where $s$ is Laplace variable.

From the definition of the wellbore storage effect:
\begin{equation*}
  q_D + C_D \frac{d p_{wD}}{d t_D} = 1
\end{equation*}
Take the Laplace transform:
\begin{equation*}
  \mathcal{L} \left[ q_D \right] + s C_D \mathcal{L} \left[ p_{wD} \right]  = \frac{1}{s}
\end{equation*}
Rearranging:
\begin{equation}
  \label{eq2}
  \mathcal{L} \left[ q_D \right]  = \frac{1}{s} - s C_D \mathcal{L} \left[ p_{wD} \right]
\end{equation}

Substituting Equation \ref{eq2} into Equation \ref{eq1}, we eliminate $\mathcal{L} \left[ q_D \right]$:
\begin{equation*}
  \mathcal{L} \left[ p_{wD} \right] = s \left( \frac{1}{s} - s C_D \mathcal{L} \left[ p_{wD} \right] \right) \mathcal{L} \left[ p_{wDc} \right]
\end{equation*}
\begin{equation*}
  \left( 1 + s^2 C_D \mathcal{L} \left[ p_{wDc} \right] \right)\mathcal{L} \left[ p_{wD} \right] = \mathcal{L} \left[ p_{wDc} \right]
\end{equation*}
\begin{equation*}
  \mathcal{L} \left[ p_{wD} \right] = \frac{\mathcal{L} \left[ p_{wDc} \right]}{1 + s^2 C_D \mathcal{L} \left[ p_{wDc} \right]}
\end{equation*}

This is a very interesting and elegant result.
It shows that if we have the constant-rate solution, we can modify it to account for wellbore storage under a variable-rate condition.

\end{document}