\documentclass[letterpaper, 12pt]{article}
\usepackage{amsmath}
\usepackage[margin=1in]{geometry}
\usepackage{adjustbox}
\usepackage{graphicx}
\usepackage[final]{pdfpages}
\usepackage{bm}
\usepackage{sectsty}
\usepackage{titlesec}
\usepackage{lipsum}
\usepackage{subcaption}
\usepackage{listings}
\usepackage{pgffor}
\usepackage{rotating}
\usepackage{xcolor}  
\usepackage{amsthm}
\usepackage{csvsimple}
\usepackage{enumitem}
\usepackage{amssymb} 
\usepackage{amsfonts}
\usepackage{tikz}
\usepackage{tikz-3dplot}
% \setlength{\tabcolsep}{0.5cm} 
\newcommand{\thickhat}[1]{\mathbf{\hat{\text{$#1$}}}}
\renewcommand{\lstlistingname}{Program}

\newtheoremstyle{custom}
  {3pt}
  {3pt}
  {\itshape}
  {} 
  {\bfseries}
  {. }  
  { }   
  {\thmname{#1} \thmnumber{#2} \thmnote{ #3}}

\theoremstyle{custom}


\newtheorem{definition}{Definition}
% \newtheorem{theorem}{Theorem}
\newtheorem*{theorem}{Theorem}
\sectionfont{\fontsize{12}{15}\selectfont}
\titleformat{\section}
{\normalfont\normalsize\bfseries}
{(\thesection)}{1em}{}

\title{Unit in Petroleum Engineering}
\author{Masaru Sawata}
\begin{document}

\maketitle
In the field of petroleum engineering, many engineers (myself included) have experienced frustration with the units used in equations.
It is often necessary to remember which units are required in each specific equation.

For example, we need to use different units for $q$ in the following two equations: the first uses STB/day, while the second uses MSCF/day.

\begin{equation*}
  p_e - p_{wf} = \frac{141.2 qB\mu}{kh} \ln \frac{r_e}{r_w}
\end{equation*}
\begin{equation*}
  p_e^2 - p_{wf}^2 = \frac{1424 q z \mu T}{kh} \ln \frac{r_e}{r_w}
\end{equation*}

It can be rather cumbersome to always check what units are assumed in each equation.

I prefer the approach of considering each variable as inherently carrying its own unit.
For example, we can think of $q$ as having a value of 10 and a unit of bbl/day.
By adopting this perspective, we don't need to worry about units when writing equations;
we only need to handle units carefully during the actual calculation.

This idea is not my own -- I encountered it in the book \textit{Thermodynamics} (in Japanese) by Haruaki Tasaki, which is one of my favorite books.

\end{document}