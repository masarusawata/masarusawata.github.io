\documentclass[letterpaper, 12pt]{article}
\usepackage{amsmath}
\usepackage[margin=1in]{geometry}
\usepackage{adjustbox}
\usepackage{graphicx}
\usepackage[final]{pdfpages}
\usepackage{bm}
\usepackage{sectsty}
\usepackage{titlesec}
\usepackage{lipsum}
\usepackage{subcaption}
\usepackage{listings}
\usepackage{pgffor}
\usepackage{rotating}
\usepackage{xcolor}  
\usepackage{amsthm}
\usepackage{csvsimple}
\usepackage{enumitem}
\usepackage{amssymb} 
\usepackage{amsfonts}
\usepackage{tikz}
\usepackage{tikz-3dplot}
% \setlength{\tabcolsep}{0.5cm} 
\newcommand{\thickhat}[1]{\mathbf{\hat{\text{$#1$}}}}
\renewcommand{\lstlistingname}{Program}

\newtheoremstyle{custom}
  {3pt}
  {3pt}
  {\itshape}
  {} 
  {\bfseries}
  {. }  
  { }   
  {\thmname{#1} \thmnumber{#2} \thmnote{ #3}}

\theoremstyle{custom}


\newtheorem{definition}{Definition}
% \newtheorem{theorem}{Theorem}
\newtheorem*{theorem}{Theorem}
\sectionfont{\fontsize{12}{15}\selectfont}
\titleformat{\section}
{\normalfont\normalsize\bfseries}
{(\thesection)}{1em}{}

\title{Understanding Volume in $n$ Dimensions Using Determinant}
\author{Masaru Sawata}
\begin{document}
\maketitle
What is the volume of a parallelepiped formed by vectors?

First, let us consider the definition of volume.
In the one-dimensional case, the ``volume" is simply the length of a vector.

What about the two-dimensional case?
We multiply the length of one vector (i.e., 1D volume) by the vertical component of the other vector.

In three dimensions, the volume is obtained by multiplying the (2D) area formed by two vectors by the component of the third vector that is perpendicular to the plane spanned by the first two.

This idea can be generalized to higher dimensions.

Suppose that the determinant of the matrix $(\vec{a}_i \cdot \vec{a}_j)$ gives the square of the volume of a parallelepiped formed by $n$ vectors $\vec{a}_i$ ($i = 1, 2, \ldots, n$). That is,

\begin{equation*}
  V_n^2 = 
  \begin{vmatrix}
    \vec{a}_1 \cdot \vec{a}_1 & \cdots & \vec{a}_1 \cdot \vec{a}_n \\
    \vdots & \ddots &  \vdots \\
    \vec{a}_n \cdot \vec{a}_1 & \cdots & \vec{a}_n \cdot \vec{a}_n
  \end{vmatrix}
\end{equation*}
where $V_n$ is the volume of the parallelepiped formed by the $n$ vectors.

Now consider the following vector $\vec{b}_{n+1}$:
\begin{equation*}
  \vec{b}_{n+1} = \frac{
    \begin{vmatrix}
    \vec{a}_1 \cdot \vec{a}_1 & \cdots & \vec{a}_1 \cdot \vec{a}_n & \vec{a}_1 \cdot \vec{a}_{n+1} \\
    \vdots & \ddots &  \vdots & \vdots\\
    \vec{a}_n \cdot \vec{a}_1 & \cdots & \vec{a}_n \cdot \vec{a}_n & \vec{a}_n \cdot \vec{a}_{n+1}\\
    \vec{a}_1 & \cdots & \vec{a}_n & \vec{a}_{n+1}
    \end{vmatrix}
  }
  {
  \begin{vmatrix}
    \vec{a}_1 \cdot \vec{a}_1 & \cdots & \vec{a}_1 \cdot \vec{a}_n \\
    \vdots & \ddots &  \vdots \\
    \vec{a}_n \cdot \vec{a}_1 & \cdots & \vec{a}_n \cdot \vec{a}_n
  \end{vmatrix}
  }
\end{equation*}

It can be easily confirmed that $\vec{a}_i \cdot \vec{b}_{n+1} = 0$ for $i = 1, 2, \cdots, n$, since each dot product makes the last row identical to the $i$-th row.

Additionally, the coefficient of $\vec{a}_i$ in the expansion is either $+1$ or $-1$.
Therefore, $\vec{b}_{n+1}$ (or $-\vec{b}_{n+1}$) represents a vector rejection--that is, the component of $\vec{a}_{n+1}$ orthogonal to the subspace spanned by $\vec{a}_1, \cdots, \vec{a}_n$.

This means we can compute the volume $V_{n+1}$ of the parallelepiped formed by the $n+1$ vectors by multiplying $V_n$ by the length $\left| \vec{b}_{n+1} \right|$:


\begin{align*}
  \left| \vec{b}_{n+1} \right|^2 
  &= \vec{b}_{n+1} \cdot \vec{b}_{n+1}\\
  &= \vec{b}_{n+1} \cdot \sum_{i=1}^{n+1}c_i \vec{a}_{i} \quad (c_i \in \mathbb{R})\\
  &= \left| \vec{b}_{n+1} \cdot \vec{a}_{n+1} \right| \\
  &=  \frac{
    \begin{vmatrix}
      \vec{a}_1 \cdot \vec{a}_1 & \cdots & \vec{a}_1 \cdot \vec{a}_n & \vec{a}_1 \cdot \vec{a}_{n+1} \\
      \vdots & \ddots &  \vdots & \vdots\\
      \vec{a}_n \cdot \vec{a}_1 & \cdots & \vec{a}_n \cdot \vec{a}_n & \vec{a}_n \cdot \vec{a}_{n+1}\\
      \vec{a}_{n+1} \cdot \vec{a}_1 & \cdots & \vec{a}_{n+1} \cdot \vec{a}_n & \vec{a}_{n+1} \cdot \vec{a}_{n+1}
    \end{vmatrix}
  }{V_n^2}
\end{align*}
The numerator of the above has to be non-negative since
\begin{align*}
  \begin{vmatrix}
      \vec{a}_1 \cdot \vec{a}_1 & \cdots & \vec{a}_1 \cdot \vec{a}_n & \vec{a}_1 \cdot \vec{a}_{n+1} \\
      \vdots & \ddots &  \vdots & \vdots\\
      \vec{a}_n \cdot \vec{a}_1 & \cdots & \vec{a}_n \cdot \vec{a}_n & \vec{a}_n \cdot \vec{a}_{n+1}\\
      \vec{a}_{n+1} \cdot \vec{a}_1 & \cdots & \vec{a}_{n+1} \cdot \vec{a}_n & \vec{a}_{n+1} \cdot \vec{a}_{n+1}
    \end{vmatrix}&=
    \begin{vmatrix}
      \vec{a}_1  & \cdots  & \vec{a}_{n+1} 
    \end{vmatrix}
    \begin{vmatrix}
      \vec{a}_1  & \cdots  & \vec{a}_{n+1} 
    \end{vmatrix}\\
    &=
    \begin{vmatrix}
      \vec{a}_1  & \cdots  & \vec{a}_{n+1} 
    \end{vmatrix}^2
\end{align*}

Therefore, in summary,
\begin{align*}
    \begin{vmatrix}
      \vec{a}_1 \cdot \vec{a}_1 & \cdots & \vec{a}_1 \cdot \vec{a}_n & \vec{a}_1 \cdot \vec{a}_{n+1} \\
      \vdots & \ddots &  \vdots & \vdots\\
      \vec{a}_n \cdot \vec{a}_1 & \cdots & \vec{a}_n \cdot \vec{a}_n & \vec{a}_n \cdot \vec{a}_{n+1}\\
      \vec{a}_{n+1} \cdot \vec{a}_1 & \cdots & \vec{a}_{n+1} \cdot \vec{a}_n & \vec{a}_{n+1} \cdot \vec{a}_{n+1}
    \end{vmatrix}
    =\left| \vec{b}_{n+1} \right|^2 V_n^2 = V_{n+1}^2
\end{align*}
Hence,
\begin{equation*}
    V_{n+1}^2 = \begin{vmatrix}
      \vec{a}_1  & \cdots  & \vec{a}_{n+1} 
    \end{vmatrix}^2
\end{equation*}

We know $V_2^2 = \left| \vec{a}_1 \, \,  \vec{a}_2 \right|^2$ gives the area of parallelogram.
Therefore,  the volume of a parallelepiped formed by $n+1$ vectors is given by the absolute value of the determinant of the matrix
 $\left| \vec{a}_1 \, \, \cdots \, \, \vec{a}_{n+1} \right|$.
Actually, the sign of the determinant tells us whether the system is right-handed or left-handed.
Perhaps I'll write about that next time.

\end{document}