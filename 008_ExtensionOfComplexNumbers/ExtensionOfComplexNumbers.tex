\documentclass[letterpaper, 12pt]{article}
\usepackage{amsmath}
\usepackage[margin=1in]{geometry}
\usepackage{adjustbox}
\usepackage{graphicx}
\usepackage[final]{pdfpages}
\usepackage{bm}
\usepackage{sectsty}
\usepackage{titlesec}
\usepackage{lipsum}
\usepackage{subcaption}
\usepackage{listings}
\usepackage{pgffor}
\usepackage{rotating}
\usepackage{xcolor}  
\usepackage{amsthm}
\usepackage{csvsimple}
\usepackage{enumitem}
\usepackage{amssymb} 
\usepackage{amsfonts}
\usepackage{tikz}
\usepackage{tikz-3dplot}
% \setlength{\tabcolsep}{0.5cm} 
\newcommand{\thickhat}[1]{\mathbf{\hat{\text{$#1$}}}}
\renewcommand{\lstlistingname}{Program}

\newtheoremstyle{custom}
  {3pt}
  {3pt}
  {\itshape}
  {} 
  {\bfseries}
  {. }  
  { }   
  {\thmname{#1} \thmnumber{#2} \thmnote{ #3}}

\theoremstyle{custom}


\newtheorem{definition}{Definition}
% \newtheorem{theorem}{Theorem}
\newtheorem*{theorem}{Theorem}
\sectionfont{\fontsize{12}{15}\selectfont}
\titleformat{\section}
{\normalfont\normalsize\bfseries}
{(\thesection)}{1em}{}

\title{The Limits of Extending the Complex Numbers}
\author{Masaru Sawata}
\begin{document}
\maketitle
A complex number is an element of the field $\mathbb{C}$.

When I was in high school, I wondered why we only cared about  $\sqrt{-1}$.
Of course, we can express $\sqrt{i}$ by a unique complex number, 
but there may be additional element, for example, $j$ to extend the number we deal with.

Of course, we can express $\sqrt{i}$ as a unique complex number,
but I thought there might be other elements--like a new symbol 
$j$--that could extend the number system we use.


At that time, I didn't realize how many useful and beautiful results can be derived simply by introducing $i$.
However, adding a new element like $j$ would destroy the structure of a field.
I came across this discussion in one of my favorite books, ``Introduction to Complex Functions" by Michio Jimbo.

Suppose there exists a field in which every number can be uniquely written as $a+bi+cj$ with $a,b,c \in \mathbb{R}$.
Then, $j^2$ must also be expressible in the same form:
\begin{equation*}
  j^2 = a + bi + cj
\end{equation*}
This leads to a quadratic equation, which means we could express $j$ using just complex numbers.
But that contradicts the assumption that $a+bi+cj$ gives a unique representation.
In other words, any $a+bi+cj$ can be expressed using $a^\prime + b^\prime i$ with $a^\prime, b^\prime \in \mathbb{R}$.

Therefore, no such field can exist.

Although we failed to add another element to the complex numbers while preserving field properties, there is an interesting historical development.
Hamilton introduced quaternions, where numbers are expressed as $a+bi+cj+dk$ with $a,b,c,d \in \mathbb{R}$.

These elements satisfy $i^2 = j^2 = k^2 = ijk = -1$.
Of course, quaternions do not form a field.

There is a fascinating fact: multiplication of quaternions encodes both the dot product and the cross product simultaneously.

In fact, there is an even larger number system called the octonions,
but as we add more elements, we lose more of the familiar properties of arithmetic operations.

\end{document}