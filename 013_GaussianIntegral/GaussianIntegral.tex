\documentclass[letterpaper, 12pt]{article}
\usepackage{amsmath}
\usepackage[margin=1in]{geometry}
\usepackage{adjustbox}
\usepackage{graphicx}
\usepackage[final]{pdfpages}
\usepackage{bm}
\usepackage{sectsty}
\usepackage{titlesec}
\usepackage{lipsum}
\usepackage{subcaption}
\usepackage{listings}
\usepackage{pgffor}
\usepackage{rotating}
\usepackage{xcolor}  
\usepackage{amsthm}
\usepackage{csvsimple}
\usepackage{enumitem}
\usepackage{amssymb} 
\usepackage{amsfonts}
\usepackage{tikz}
\usepackage{tikz-3dplot}
% \setlength{\tabcolsep}{0.5cm} 
\newcommand{\thickhat}[1]{\mathbf{\hat{\text{$#1$}}}}
\renewcommand{\lstlistingname}{Program}

\newtheoremstyle{custom}
  {3pt}
  {3pt}
  {\itshape}
  {} 
  {\bfseries}
  {. }  
  { }   
  {\thmname{#1} \thmnumber{#2} \thmnote{ #3}}

\theoremstyle{custom}


\newtheorem{definition}{Definition}
% \newtheorem{theorem}{Theorem}
\newtheorem*{theorem}{Theorem}
\sectionfont{\fontsize{12}{15}\selectfont}
\titleformat{\section}
{\normalfont\normalsize\bfseries}
{(\thesection)}{1em}{}

\title{Gaussian Integral}
\date{}
\begin{document}
\maketitle
Sometimes, we need to evaluate the following integral:
\begin{equation*}
  \int_{-\infty}^{\infty} e^{-x^2} \, dx
\end{equation*}
Surprisingly, this is equal to $\sqrt{\pi}$.

There are several ways to evaluate this integral, but an elegant and simple method is shown here.

Let $I$ be the value of the integral:
\begin{equation*}
  I = \int_{-\infty}^{\infty} e^{-x^2} \, dx
\end{equation*}
We consider $I^2$:
\begin{align*}
  I^2 
  &= \left( \int_{-\infty}^{\infty} e^{-x^2} \, dx \right)^2 \\
  &= \int_{-\infty}^{\infty} e^{-x^2} \, dx \int_{-\infty}^{\infty} e^{-x^2} \, dx \\
  &= \int_{-\infty}^{\infty} e^{-x^2} \, dx \int_{-\infty}^{\infty} e^{-y^2} \, dy \\
  &= \int_{-\infty}^{\infty} \int_{-\infty}^{\infty} e^{-(x^2+y^2)} \, dx \, dy
\end{align*}
Then, Next, we change variables from Cartesian coordinates $(x,y)$ to polar coordinates $(r, \theta)$:
\begin{align*}
    I^2 
  &= \int_{-\infty}^{\infty} \int_{-\infty}^{\infty} e^{-(x^2+y^2)} \, dx \, dy \\
  &= \int_{0}^{2\pi} \int_{0}^{\infty} e^{-r^2} r \, dr \, d\theta \\
  &= 2\pi \int_{0}^{\infty} re^{-r^2} \, dr \\
  &= 2\pi \left[ -\frac{1}{2}e^{-r^2} \right]_{0}^{\infty}\\
  &= \pi
\end{align*}
Since the integrand is always positive ($e^{-x^2}>0$), we conclude that $I>0$. Therefore,
\begin{equation*}
  \int_{-\infty}^{\infty} e^{-x^2} \, dx = I = \sqrt{\pi}
\end{equation*}

\end{document}