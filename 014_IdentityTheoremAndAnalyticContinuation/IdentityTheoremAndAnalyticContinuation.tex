\documentclass[letterpaper, 12pt]{article}
\usepackage{amsmath}
\usepackage[margin=1in]{geometry}
\usepackage{adjustbox}
\usepackage{graphicx}
\usepackage[final]{pdfpages}
\usepackage{bm}
\usepackage{sectsty}
\usepackage{titlesec}
\usepackage{lipsum}
\usepackage{subcaption}
\usepackage{listings}
\usepackage{pgffor}
\usepackage{rotating}
\usepackage{xcolor}  
\usepackage{amsthm}
\usepackage{csvsimple}
\usepackage{enumitem}
\usepackage{amssymb} 
\usepackage{amsfonts}
\usepackage{tikz}
\usepackage{tikz-3dplot}
% \setlength{\tabcolsep}{0.5cm} 
\newcommand{\thickhat}[1]{\mathbf{\hat{\text{$#1$}}}}
\renewcommand{\lstlistingname}{Program}

\newtheoremstyle{custom}
  {3pt}
  {3pt}
  {\itshape}
  {} 
  {\bfseries}
  {. }  
  { }   
  {\thmname{#1} \thmnumber{#2} \thmnote{ #3}}

\theoremstyle{custom}


\newtheorem{definition}{Definition}
% \newtheorem{theorem}{Theorem}
\newtheorem*{theorem}{}
\sectionfont{\fontsize{12}{15}\selectfont}
\titleformat{\section}
{\normalfont\normalsize\bfseries}
{(\thesection)}{1em}{}

\title{Identity Theorem and Analytic Continuation}
\date{}
\begin{document}
\maketitle
\begin{theorem}[Identity Theorem]
    Suppose $f(z)$ and $g(z)$ are analytic on a domain $\Omega$,
    and that a sequence of points $z_n \in \Omega$, with $z_n \neq a$ for all $n$, converges to a point $a \in \Omega$.
    If $f(z_n) = g(z_n)$ for all $n=1,2,3,\cdots$, then $f(z) = g(z)$ for all $z \in \Omega$.
\end{theorem}

Let $F(z) = f(z) - g(z)$.
Since $f(z)$ and $g(z)$ are analytic on $\Omega$, $F(z)$ is also analytic and can be represented by a Taylor series expansion about $a$:
\begin{equation*}
  F(z) = \sum_{n=0}^{\infty} c_n (z-a)^n
\end{equation*}
At each point $z_n$, we have $F(z_n) = 0$.
By analyticity, $F(z)$ is continuous at $a$, so
\begin{equation*}
  \lim_{n\rightarrow \infty} F(z_n) = F(a) = 0
\end{equation*}
Thus, $c_0 = 0$.


Next, define $F_1(z) = \frac{F(z)}{z-a}$. (Note that $F_1(z)$ is analytic in a neighborhood of $a$ and has the same radius of convergence as $F(z)$.)
Applying the same argument, we see that $c_1 = 0$.
Continuing this process inductively, we conclude that $c_n = 0$ for all $n$.
Therefore, $F(z)$ is identically zero in the neighborhood of $a$.

Now, choose an arbitrary point $b \in \Omega$ and connect it to $a$ by a curve lying entirely within $\Omega$, parameterized as $z(t)$ with $z(0)=a$ and $z(1)=b$.
Suppose, for contradiction, that there exists a maximal $t_0 < 1$ such that $F(z(t)) = 0$ for all $0 \leq t \leq t_0$.
Consider a sequence $z_n = z(t_n)$, where $t_n \to t_0$ from below.
Since $F(z_n) = 0$ and $F$ is analytic, its Taylor expansion at $z(t_0)$ must also vanish identically, and $F(z) = 0$ in a neighborhood of $z(t_0)$.
This contradicts the maximality of $t_0$.
Therefore, $F(z) = 0$ along the entire curve, and since $b$ was arbitrary, $F(z) \equiv 0$ on $\Omega$.
Hence, $f(z) = g(z)$ on $\Omega$.\\


From the Identity Theorem, if there is an analytic function $f_1(z)$ on a domain $\Omega_1$, and another analytic function $f_2(z)$ on a larger domain $\Omega_2 \supset \Omega_1$, with $f_1(z) = f_2(z)$ on 
$\Omega_1$, then $f_2(z)$ is uniquely determined on $\Omega_2$; this extension is called analytic continuation.
Thus, extending the domain allows us to study the global properties of the function.

It is well known that the trigonometric and exponential functions are closely related on the complex plane.
However, if we consider their behavior only on the real domain, we are restricting ourselves to just a part of the function.

This proof closely follows the very insightful book, ``Introduction to Complex Function Theory'' (in Japanese), by Michio Jimbo.

\end{document}