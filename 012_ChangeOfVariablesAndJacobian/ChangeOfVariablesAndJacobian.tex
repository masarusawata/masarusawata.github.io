\documentclass[letterpaper, 12pt]{article}
\usepackage{amsmath}
\usepackage[margin=1in]{geometry}
\usepackage{adjustbox}
\usepackage{graphicx}
\usepackage[final]{pdfpages}
\usepackage{bm}
\usepackage{sectsty}
\usepackage{titlesec}
\usepackage{lipsum}
\usepackage{subcaption}
\usepackage{listings}
\usepackage{pgffor}
\usepackage{rotating}
\usepackage{xcolor}  
\usepackage{amsthm}
\usepackage{csvsimple}
\usepackage{enumitem}
\usepackage{amssymb} 
\usepackage{amsfonts}
\usepackage{tikz}
\usepackage{tikz-3dplot}
% \setlength{\tabcolsep}{0.5cm} 
\newcommand{\thickhat}[1]{\mathbf{\hat{\text{$#1$}}}}
\renewcommand{\lstlistingname}{Program}

\newtheoremstyle{custom}
  {3pt}
  {3pt}
  {\itshape}
  {} 
  {\bfseries}
  {. }  
  { }   
  {\thmname{#1} \thmnumber{#2} \thmnote{ #3}}

\theoremstyle{custom}


\newtheorem{definition}{Definition}
% \newtheorem{theorem}{Theorem}
\newtheorem*{theorem}{Theorem}
\sectionfont{\fontsize{12}{15}\selectfont}
\titleformat{\section}
{\normalfont\normalsize\bfseries}
{(\thesection)}{1em}{}

\title{Understanding Change of Variables in Integration: Why the Jacobian Determinant Appears}
\author{Masaru Sawata}
\begin{document}
\maketitle

When we change variables in an integral, we need to multiply the integrand by the absolute value of the determinant of the Jacobian matrix.
There is an intuitive way to understand why this is the case.

Let us consider a change of variables from $(x_1, x_2, \cdots , x_n)$ to $(y_1, y_2, \cdots , y_n)$.
The volume of hyperrectangle in the new ($y$) coordinates, denoted by $\Delta V_y$ is given by $\displaystyle \prod_{i=1}^n \Delta y_i$, where $\Delta y_i > 0$.

Each edge vector of the hyperrectangle in the $y$-coordinates corresponds to a vector $\vec{v}_i$ in the original $x$-coordinates:
\begin{equation*}
  \vec{v}_i = \frac{\partial \vec{x}}{\partial y_i} \Delta y_i + \vec{\varepsilon}_i
\end{equation*}
where $\vec{\varepsilon}_i$ satisfies
\begin{equation*}  
  \lim_{\Delta y_i \rightarrow 0} \frac{\vec{\varepsilon}_i}{\Delta y_i} = 0
\end{equation*}

The volume $\Delta V_x$ of the corresponding parallelotope in the original coordinate system is:
\begin{align*}
  \Delta V_x &= \left| \det
  \begin{pmatrix}
    \displaystyle \frac{\partial \vec{x}}{\partial y_1} \Delta y_1 + \vec{\varepsilon}_1 & \cdots & \displaystyle \frac{\partial \vec{x}}{\partial y_n} \Delta y_n + \vec{\varepsilon}_n
  \end{pmatrix}\right| \\
  &=\left| \det 
  \begin{pmatrix}
    \displaystyle \frac{\partial \vec{x}}{\partial y_1} + \frac{\vec{\varepsilon}_1}{\Delta y_1} & \cdots & \displaystyle \frac{\partial \vec{x}}{\partial y_n} + \frac{\vec{\varepsilon}_n}{\Delta y_n} 
  \end{pmatrix}\right|
  \prod_{i=1}^n \Delta y_i \\
  &= \left| \det \begin{pmatrix}
    \displaystyle \frac{\partial \vec{x}}{\partial y_1} + \frac{\vec{\varepsilon}_1}{\Delta y_1} & \cdots & \displaystyle \frac{\partial \vec{x}}{\partial y_n} + \frac{\vec{\varepsilon}_n}{\Delta y_n} 
  \end{pmatrix}\right| \Delta V_y
\end{align*}

Hence, the volume ratio is given by:
\begin{equation*}
  \left| \det 
  \begin{pmatrix}
    \displaystyle \frac{\partial \vec{x}}{\partial y_1} + \frac{\vec{\varepsilon}_1}{\Delta y_1} & \cdots & \displaystyle \frac{\partial \vec{x}}{\partial y_n} + \frac{\vec{\varepsilon}_n}{\Delta y_n} 
  \end{pmatrix} \right|
  \rightarrow \left| \det 
  \begin{pmatrix}
    \displaystyle \frac{\partial \vec{x}}{\partial y_1} & \cdots & \displaystyle \frac{\partial \vec{x}}{\partial y_n} 
  \end{pmatrix} \right| \quad \text{as } \Delta y_i \rightarrow 0 \text{ for all } i
\end{equation*}

This is (the absolute value of) the determinant of the transpose of the Jacobian matrix.
Since the determinant is unchanged under transposition, this is equal to the determinant of the Jacobian matrix itself.




\end{document}