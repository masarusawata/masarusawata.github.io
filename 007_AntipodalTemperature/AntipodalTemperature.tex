\documentclass[letterpaper, 12pt]{article}
\usepackage{amsmath}
\usepackage[margin=1in]{geometry}
\usepackage{adjustbox}
\usepackage{graphicx}
\usepackage[final]{pdfpages}
\usepackage{bm}
\usepackage{sectsty}
\usepackage{titlesec}
\usepackage{lipsum}
\usepackage{subcaption}
\usepackage{listings}
\usepackage{pgffor}
\usepackage{rotating}
\usepackage{xcolor}  
\usepackage{amsthm}
\usepackage{csvsimple}
\usepackage{enumitem}
\usepackage{amssymb} 
\usepackage{amsfonts}
\usepackage{tikz}
\usepackage{tikz-3dplot}
% \setlength{\tabcolsep}{0.5cm} 
\newcommand{\thickhat}[1]{\mathbf{\hat{\text{$#1$}}}}
\renewcommand{\lstlistingname}{Program}

\newtheoremstyle{custom}
  {3pt}
  {3pt}
  {\itshape}
  {} 
  {\bfseries}
  {. }  
  { }   
  {\thmname{#1} \thmnumber{#2} \thmnote{ #3}}

\theoremstyle{custom}


\newtheorem{definition}{Definition}
% \newtheorem{theorem}{Theorem}
\newtheorem*{theorem}{Theorem}
\sectionfont{\fontsize{12}{15}\selectfont}
\titleformat{\section}
{\normalfont\normalsize\bfseries}
{(\thesection)}{1em}{}

\title{Somewhere on Earth, Antipodal Points Must Share a Temperature}
\author{Masaru Sawata}
\begin{document}
\maketitle
There must be a pair of opposite points on Earth that are equally warm.
This fact can be proven under a simple assumption: that temperature varies continuously over the Earth's surface.

Let temperature be a function of position, denoted $f:S^2 \rightarrow \mathbb{R}$, where $S^2$ represents the surface of a sphere.
Now, consider the function
\begin{equation*}
  g(\vec{x}) = f(\vec{x}) - f(-\vec{x})
\end{equation*}
This function represents the temperature difference between a point and its antipode.
Clearly, $g$ is an odd function, meaning $g(\vec{x}) = -g(-\vec{x})$.

Suppose that for some point $\vec{x_0}$, we have $g(\vec{x_0}) > 0$. 
(If $g(\vec{x_0}) < 0$, we simply let $\vec{x_0}$ be $-\vec{x_0}$ instead. 
If $g(\vec{x_0}) = 0$, then $f(\vec{x}) = f(-\vec{x})$, meaning the temperatures already match!)
Then, by the oddness of $g$, we have $g(-\vec{x_0}) = -g(\vec{x_0})< 0$
Now, consider any continuous path on the sphere connecting $\vec{x_0}$ and $-\vec{x_0}$.

By the Intermediate Value Theorem, there must exist a point along this path where $g(\vec{x})=0$.
That is, $f(\vec{x}) = f(-\vec{x})$: the temperature at some point on Earth must be equal to that at its antipode.

This result is a special case of the Borsuk--Ulam Theorem, which more generally states that any continuous function from an
$n$-sphere to $\mathbb{R}^n$ maps some pair of antipodal points to the same value.

In fact, using this theorem, we can show that there exists a pair of antipodal points on Earth with not only the same temperature, but also the same pressure.
Maybe I'll write about that another time.

\end{document}