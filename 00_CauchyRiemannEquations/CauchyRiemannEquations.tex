\documentclass[letterpaper, 12pt]{article}
\usepackage{amsmath}
\usepackage[margin=1in]{geometry}
\usepackage{adjustbox}
\usepackage{graphicx}
\usepackage[final]{pdfpages}
\usepackage{bm}
\usepackage{sectsty}
\usepackage{titlesec}
\usepackage{lipsum}
\usepackage{subcaption}
\usepackage{listings}
\usepackage{pgffor}
\usepackage{rotating}
\usepackage{xcolor}  
\usepackage{amsthm}
\usepackage{csvsimple}
\usepackage{enumitem}
\usepackage{amssymb} 
\usepackage{amsfonts}
\usepackage{tikz}
\usepackage{tikz-3dplot}
% \setlength{\tabcolsep}{0.5cm} 
\newcommand{\thickhat}[1]{\mathbf{\hat{\text{$#1$}}}}
\renewcommand{\lstlistingname}{Program}

\newtheoremstyle{custom}
  {3pt}
  {3pt}
  {\itshape}
  {} 
  {\bfseries}
  {. }  
  { }   
  {\thmname{#1} \thmnumber{#2} \thmnote{ #3}}

\theoremstyle{custom}


\newtheorem{definition}{Definition}
% \newtheorem{theorem}{Theorem}
\newtheorem*{theorem}{Theorem}
\sectionfont{\fontsize{12}{15}\selectfont}
\titleformat{\section}
{\normalfont\normalsize\bfseries}
{(\thesection)}{1em}{}

\title{Cauchy--Riemann Equations}
\author{Masaru Sawata}
\begin{document}

\maketitle

\begin{theorem}[Cauchy--Riemann Equations]
\end{theorem}
If a complex-valued function $f: \mathbb{C} \rightarrow \mathbb{C}$ is analytic at $z(=x+iy)$, then the following equations hold. ($x\in \mathbb{R}$ and $y\in \mathbb{R}$)
\begin{align*}
  \frac{\partial u}{\partial x} &= \frac{\partial v}{\partial y} \\
  \frac{\partial u}{\partial y} &= -\frac{\partial v}{\partial x}
\end{align*}
where $u: \mathbb{R}^2 \rightarrow \mathbb{R}$ and $v: \mathbb{R}^2 \rightarrow \mathbb{R}$ are real part and imaginary part of $f$.
This means that $f(z) = f(x+iy) = u(x,y) + iv(x,y)$\\
\begin{proof}
  Since $f$ is analytic at $z$, there exists $\alpha \in \mathbb{C}$ which satisfies,
  \begin{equation*}
    f(z+\Delta z) - f(z) = \alpha \Delta z + o(\Delta z)
  \end{equation*}
  where $o$ is a Landau's little-o notation.
  (This means that $\alpha$ does not change by how $\Delta z$ approaches to zero.)

  If we write this using $u$ and $v$, we obtain
  \begin{align*}
    &\left( u(x+\Delta x, y+\Delta y) + iv(x+\Delta x, y+\Delta y) \right) - \left( u(x, y) + iv(x, y) \right) \\
    & \quad = \alpha \left( \Delta x + i\Delta y \right) + o\left( \sqrt{\Delta x^2 + \Delta y^2}\right)
  \end{align*}
  
  This equation holds even if we consider $\Delta y=0$ and let $\Delta x$ approach to zero.
  Hence,
  \begin{equation*}
    \left( u(x+\Delta x, y) + iv(x+\Delta x, y) \right) - \left( u(x, y) + iv(x, y) \right)
    = \alpha \Delta x + o\left( \Delta x \right)
  \end{equation*}
  ($o(\left| \Delta x \right|)$ is equivalent to $o(\Delta x)$)\\
  
  Therefore,
  \begin{equation*}
    \frac{u(x+\Delta x, y) - u(x, y)}{\Delta x} + i \frac{v(x+\Delta x, y) - v(x, y)}{\Delta x} = \alpha + \frac{o\left( \Delta x \right)}{\Delta x}
  \end{equation*}
  As $\Delta x \rightarrow 0$,
  \begin{equation*}
    \frac{\partial u}{\partial x} + i \frac{\partial v}{\partial x} = \alpha
  \end{equation*}

  Similarly, we consider $\Delta x=0$ and make $\Delta y$ approach to zero.
  \begin{equation*}
    \left( u(x, y+\Delta y) + iv(x, y+\Delta y) \right) - \left( u(x, y) + iv(x, y) \right)
    = i \alpha \Delta y + o\left( \Delta y \right)
  \end{equation*}
 
  Therefore,
  \begin{equation*}
    -i \frac{u(x, y+\Delta y) - u(x, y)}{\Delta y} + \frac{v(x, y+\Delta y) - v(x, y)}{\Delta y} = \alpha + \frac{o\left( \Delta y \right)}{\Delta y}
  \end{equation*}
  As $\Delta y \rightarrow 0$,
  \begin{equation*}
    -i\frac{\partial u}{\partial y} +  \frac{\partial v}{\partial y} = \alpha
  \end{equation*}
  
  By comparing $\alpha$ obtained by the above, we obtain
  \begin{align*}
    \frac{\partial u}{\partial x} &= \frac{\partial v}{\partial y} \\
    \frac{\partial u}{\partial y} &= -\frac{\partial v}{\partial x}
  \end{align*}
\end{proof}


\end{document}